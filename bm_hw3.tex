\documentclass[]{article}
\usepackage{lmodern}
\usepackage{amssymb,amsmath}
\usepackage{ifxetex,ifluatex}
\usepackage{fixltx2e} % provides \textsubscript
\ifnum 0\ifxetex 1\fi\ifluatex 1\fi=0 % if pdftex
  \usepackage[T1]{fontenc}
  \usepackage[utf8]{inputenc}
\else % if luatex or xelatex
  \ifxetex
    \usepackage{mathspec}
  \else
    \usepackage{fontspec}
  \fi
  \defaultfontfeatures{Ligatures=TeX,Scale=MatchLowercase}
\fi
% use upquote if available, for straight quotes in verbatim environments
\IfFileExists{upquote.sty}{\usepackage{upquote}}{}
% use microtype if available
\IfFileExists{microtype.sty}{%
\usepackage{microtype}
\UseMicrotypeSet[protrusion]{basicmath} % disable protrusion for tt fonts
}{}
\usepackage[margin=1in]{geometry}
\usepackage{hyperref}
\hypersetup{unicode=true,
            pdftitle={Biostatistical Methods Homework 3},
            pdfborder={0 0 0},
            breaklinks=true}
\urlstyle{same}  % don't use monospace font for urls
\usepackage{color}
\usepackage{fancyvrb}
\newcommand{\VerbBar}{|}
\newcommand{\VERB}{\Verb[commandchars=\\\{\}]}
\DefineVerbatimEnvironment{Highlighting}{Verbatim}{commandchars=\\\{\}}
% Add ',fontsize=\small' for more characters per line
\usepackage{framed}
\definecolor{shadecolor}{RGB}{248,248,248}
\newenvironment{Shaded}{\begin{snugshade}}{\end{snugshade}}
\newcommand{\KeywordTok}[1]{\textcolor[rgb]{0.13,0.29,0.53}{\textbf{#1}}}
\newcommand{\DataTypeTok}[1]{\textcolor[rgb]{0.13,0.29,0.53}{#1}}
\newcommand{\DecValTok}[1]{\textcolor[rgb]{0.00,0.00,0.81}{#1}}
\newcommand{\BaseNTok}[1]{\textcolor[rgb]{0.00,0.00,0.81}{#1}}
\newcommand{\FloatTok}[1]{\textcolor[rgb]{0.00,0.00,0.81}{#1}}
\newcommand{\ConstantTok}[1]{\textcolor[rgb]{0.00,0.00,0.00}{#1}}
\newcommand{\CharTok}[1]{\textcolor[rgb]{0.31,0.60,0.02}{#1}}
\newcommand{\SpecialCharTok}[1]{\textcolor[rgb]{0.00,0.00,0.00}{#1}}
\newcommand{\StringTok}[1]{\textcolor[rgb]{0.31,0.60,0.02}{#1}}
\newcommand{\VerbatimStringTok}[1]{\textcolor[rgb]{0.31,0.60,0.02}{#1}}
\newcommand{\SpecialStringTok}[1]{\textcolor[rgb]{0.31,0.60,0.02}{#1}}
\newcommand{\ImportTok}[1]{#1}
\newcommand{\CommentTok}[1]{\textcolor[rgb]{0.56,0.35,0.01}{\textit{#1}}}
\newcommand{\DocumentationTok}[1]{\textcolor[rgb]{0.56,0.35,0.01}{\textbf{\textit{#1}}}}
\newcommand{\AnnotationTok}[1]{\textcolor[rgb]{0.56,0.35,0.01}{\textbf{\textit{#1}}}}
\newcommand{\CommentVarTok}[1]{\textcolor[rgb]{0.56,0.35,0.01}{\textbf{\textit{#1}}}}
\newcommand{\OtherTok}[1]{\textcolor[rgb]{0.56,0.35,0.01}{#1}}
\newcommand{\FunctionTok}[1]{\textcolor[rgb]{0.00,0.00,0.00}{#1}}
\newcommand{\VariableTok}[1]{\textcolor[rgb]{0.00,0.00,0.00}{#1}}
\newcommand{\ControlFlowTok}[1]{\textcolor[rgb]{0.13,0.29,0.53}{\textbf{#1}}}
\newcommand{\OperatorTok}[1]{\textcolor[rgb]{0.81,0.36,0.00}{\textbf{#1}}}
\newcommand{\BuiltInTok}[1]{#1}
\newcommand{\ExtensionTok}[1]{#1}
\newcommand{\PreprocessorTok}[1]{\textcolor[rgb]{0.56,0.35,0.01}{\textit{#1}}}
\newcommand{\AttributeTok}[1]{\textcolor[rgb]{0.77,0.63,0.00}{#1}}
\newcommand{\RegionMarkerTok}[1]{#1}
\newcommand{\InformationTok}[1]{\textcolor[rgb]{0.56,0.35,0.01}{\textbf{\textit{#1}}}}
\newcommand{\WarningTok}[1]{\textcolor[rgb]{0.56,0.35,0.01}{\textbf{\textit{#1}}}}
\newcommand{\AlertTok}[1]{\textcolor[rgb]{0.94,0.16,0.16}{#1}}
\newcommand{\ErrorTok}[1]{\textcolor[rgb]{0.64,0.00,0.00}{\textbf{#1}}}
\newcommand{\NormalTok}[1]{#1}
\usepackage{longtable,booktabs}
\usepackage{graphicx,grffile}
\makeatletter
\def\maxwidth{\ifdim\Gin@nat@width>\linewidth\linewidth\else\Gin@nat@width\fi}
\def\maxheight{\ifdim\Gin@nat@height>\textheight\textheight\else\Gin@nat@height\fi}
\makeatother
% Scale images if necessary, so that they will not overflow the page
% margins by default, and it is still possible to overwrite the defaults
% using explicit options in \includegraphics[width, height, ...]{}
\setkeys{Gin}{width=\maxwidth,height=\maxheight,keepaspectratio}
\IfFileExists{parskip.sty}{%
\usepackage{parskip}
}{% else
\setlength{\parindent}{0pt}
\setlength{\parskip}{6pt plus 2pt minus 1pt}
}
\setlength{\emergencystretch}{3em}  % prevent overfull lines
\providecommand{\tightlist}{%
  \setlength{\itemsep}{0pt}\setlength{\parskip}{0pt}}
\setcounter{secnumdepth}{0}
% Redefines (sub)paragraphs to behave more like sections
\ifx\paragraph\undefined\else
\let\oldparagraph\paragraph
\renewcommand{\paragraph}[1]{\oldparagraph{#1}\mbox{}}
\fi
\ifx\subparagraph\undefined\else
\let\oldsubparagraph\subparagraph
\renewcommand{\subparagraph}[1]{\oldsubparagraph{#1}\mbox{}}
\fi

%%% Use protect on footnotes to avoid problems with footnotes in titles
\let\rmarkdownfootnote\footnote%
\def\footnote{\protect\rmarkdownfootnote}

%%% Change title format to be more compact
\usepackage{titling}

% Create subtitle command for use in maketitle
\newcommand{\subtitle}[1]{
  \posttitle{
    \begin{center}\large#1\end{center}
    }
}

\setlength{\droptitle}{-2em}

  \title{Biostatistical Methods Homework 3}
    \pretitle{\vspace{\droptitle}\centering\huge}
  \posttitle{\par}
    \author{}
    \preauthor{}\postauthor{}
    \date{}
    \predate{}\postdate{}
  

\begin{document}
\maketitle

\section{Problem 1}\label{problem-1}

\subsection{1}\label{section}

\$\$

\begin{aligned}
S^{2} &= \frac{1}{n-1}\sum_{i=1}^{n}\left(X_{i}^{2}+\bar{X}^{2}-2X_{i}\bar{X}\right)\\
&= \frac{1}{n-1}\left(\sum_{i=1}^{n}X_{i}^{2}+\sum_{i=1}^{n}\bar{X}^{2}-\sum_{i=1}^{n}2X_{i}\bar{X}\right)\\
&=\frac{1}{n-1}\left(\sum_{i=1}^{n}X_{i}^{2} + n\bar{X}^{2} - 2\bar{X}\sum_{i=1}^{n}X_{i}\right)\\
&= \frac{1}{n-1}\left(\sum_{i=1}^{n}X_{i}^{2} + n\bar{X}^{2} - 2\bar{X}n\bar{X}\right)\\
&= \frac{1}{n-1}\left(\sum_{i=1}^{n}X_{i}^{2} + n\bar{X}^{2} - 2n\bar{X}^{2}\right)\\
&= = \frac{1}{n-1}\left(\sum_{i=1}^{n}X_{i}^{2} - n\bar{X}^{2} \right)\\




\end{aligned}

\$\$

\$\$

\begin{aligned}
E(S^{2}) &= \frac{1}{n-1}\left [ E(\sum_{i=1}^{n}X_{i}^{2} )- E(n\bar{X}^{2}) \right ]\\
&= \frac{1}{n-1}\left [ nE(X_{i}^{2} )- nE(\bar{X}^{2}) \right ]\\
&= \frac{n}{n-1}\left [ E(X_{i}^{2} )- E(\bar{X}^{2}) \right ]\\

\end{aligned}

\$\$

\$\$

\begin{aligned}
Var(\bar{X}) &= E\bar{X}^{2}-(E\bar{X})^{2}


\end{aligned}

\$\$

\[
\begin{aligned}
E\bar{X}^{2} &= Var(\bar{X}) + (E\bar{X})^{2} \\
&= Var\left [ \frac{\sum_{i=1}^{n}X_{i}}{n} \right ]+\mu^{2}\\
&=\frac{1}{n^{2}}Var\left ( \sum_{i=1}^{n}X_{i} \right )+\mu^{2}\\
&=\frac{1}{n^{2}}\sum_{i=1}^{n}Var(X_{i})+\mu^{2}\\
&=\frac{1}{n^{2}}n\sigma^{2}+\mu^{2}\\
&=\frac{1}{n}\sigma^{2}+\mu^{2}\\
\end{aligned}
\]

\[
\begin{aligned}
Var(X_{i})&=EX_{i}^{2}-(EX_{i})^{2}\\
EX_{i}^{2}&=\sigma^{2}+\mu^{2}\\
E(S^{2})&=\frac{n}{n-1}\left[\sigma^{2}+\mu^{2}-(\frac{1}{n}\sigma^{2}+\mu^{2})\right]\\
&=\frac{n}{n-1}\left[\sigma^{2}-\frac{1}{n}\sigma^{2}\right]\\
&=\frac{n}{n-1}\left[\frac{n-1}{n}\sigma^{2}\right]\\
&=\sigma^{2}
\end{aligned}
\]

\subsection{2}\label{section-1}

\[
\begin{aligned}
(y_{ij}-\bar{y})^{2}&=(y_{ij}-\bar{y_{i}}+\bar{y_{i}}-\bar{y})^{2}\\
&=(y_{ij}-\bar{y_{i}})^{2}+(\bar{y_{i}}-\bar{y})^{2}-2(y_{ij}-\bar{y_{i}})(\bar{y_{i}}-\bar{y})
\end{aligned}
\]

\section{Problem 2}\label{problem-2}

Cigarette smoking continues to be a public health problem with major
consequences on heart and lung diseases. Less is actually known about
the consequences of quitting smoking. A recent study selected a group of
10 women working at a small medical practice, ages 50-64, that had
smoked at least 1 pack/day and quit for at least 6 years (data
``HeavySmoke.csv'').

\subsection{1. The first question is to assess if their body mass index
(BMI) has changed 6 years after quitting smoking. Perform an appropriate
hypothesis test and interpret your findings.
(5p)}\label{the-first-question-is-to-assess-if-their-body-mass-index-bmi-has-changed-6-years-after-quitting-smoking.-perform-an-appropriate-hypothesis-test-and-interpret-your-findings.-5p}

\begin{Shaded}
\begin{Highlighting}[]
\NormalTok{smoke_data =}\StringTok{ }\KeywordTok{read_csv}\NormalTok{(}\StringTok{"./HeavySmoke.csv"}\NormalTok{) }\OperatorTok
\StringTok{  }\NormalTok{janitor}\OperatorTok{::}\KeywordTok{clean_names}\NormalTok{() }\OperatorTok
\StringTok{  }\KeywordTok{mutate}\NormalTok{(}\DataTypeTok{diff =}\NormalTok{ bmi_base }\OperatorTok{-}\StringTok{ }\NormalTok{bmi_6yrs)}
\end{Highlighting}
\end{Shaded}

\begin{verbatim}
## Parsed with column specification:
## cols(
##   ID = col_integer(),
##   BMI_base = col_double(),
##   BMI_6yrs = col_double()
## )
\end{verbatim}

\begin{Shaded}
\begin{Highlighting}[]
\NormalTok{diff_mean =}\StringTok{ }\KeywordTok{mean}\NormalTok{(smoke_data}\OperatorTok{$}\NormalTok{diff)}
\NormalTok{diff_sd =}\StringTok{ }\KeywordTok{sd}\NormalTok{(smoke_data}\OperatorTok{$}\NormalTok{diff)}
\NormalTok{n =}\StringTok{ }\DecValTok{10}
\NormalTok{t =}\StringTok{ }\NormalTok{(diff_mean }\OperatorTok{-}\StringTok{ }\DecValTok{0}\NormalTok{)}\OperatorTok{/}\NormalTok{(diff_sd}\OperatorTok{/}\KeywordTok{sqrt}\NormalTok{(n))}
\end{Highlighting}
\end{Shaded}

\begin{Shaded}
\begin{Highlighting}[]
\KeywordTok{qt}\NormalTok{(}\FloatTok{0.975}\NormalTok{, n}\OperatorTok{-}\DecValTok{1}\NormalTok{)}
\end{Highlighting}
\end{Shaded}

\begin{verbatim}
## [1] 2.262157
\end{verbatim}

\begin{Shaded}
\begin{Highlighting}[]
\KeywordTok{t.test}\NormalTok{(smoke_data}\OperatorTok{$}\NormalTok{bmi_base, smoke_data}\OperatorTok{$}\NormalTok{bmi_6yrs, }\DataTypeTok{paired =} \OtherTok{TRUE}\NormalTok{)}
\end{Highlighting}
\end{Shaded}

\begin{verbatim}
## 
##  Paired t-test
## 
## data:  smoke_data$bmi_base and smoke_data$bmi_6yrs
## t = -4.3145, df = 9, p-value = 0.001949
## alternative hypothesis: true difference in means is not equal to 0
## 95 percent confidence interval:
##  -5.121709 -1.598291
## sample estimates:
## mean of the differences 
##                   -3.36
\end{verbatim}

\[
H_{0} : \mu_{1} - \mu_{2} = 0
\]

\[
H_{1} : \mu_{1} - \mu_{2} > 0
\]

\[
\bar{d}= \sum_{i=1}^{n}\frac{d_{i}}{n}=-3.36
\]

\[
s_{d}=\sqrt{\frac{\sum_{i=1}^{n}(d_{i}-\bar{d})^{2}}{n-1}}=2.4627
\]

\[
n = 10\\
\]

\[
t=\frac{\bar{d}-0}{s_{d}/\sqrt{n}}=-4.3145
\]

\[
t_{n-1, 1-\alpha /2}=2.262157
\]

\[
\left | t \right |=4.3145
\]

\[
For\ \left | t \right |> t_{n-1, 1-\alpha /2},\ reject\ H_{0}
\] Intepretation: We use paired t-test to test whether those 10 women's
BMI has changed over 6 years after quitting smoking. According to the
solutions listed above, we should reject the null, which means their BMI
has changed significantly over 6 years.

\subsection{2. The investigators suspected an overall change in weight
over the years, so they decided to enroll a control group of 50-64 years
of age that never smoked (data NeverSmoke.csv). Perform an appropriate
test to compare the BMI changes between women that quit smoking and
women who never smoked. Interpret the findings.
(5p)}\label{the-investigators-suspected-an-overall-change-in-weight-over-the-years-so-they-decided-to-enroll-a-control-group-of-50-64-years-of-age-that-never-smoked-data-neversmoke.csv.-perform-an-appropriate-test-to-compare-the-bmi-changes-between-women-that-quit-smoking-and-women-who-never-smoked.-interpret-the-findings.-5p}

\begin{Shaded}
\begin{Highlighting}[]
\NormalTok{nonsmoke_data =}\StringTok{ }\KeywordTok{read_csv}\NormalTok{(}\StringTok{"./NeverSmoke.csv"}\NormalTok{) }\OperatorTok
\StringTok{  }\NormalTok{janitor}\OperatorTok{::}\KeywordTok{clean_names}\NormalTok{() }\OperatorTok
\StringTok{  }\KeywordTok{mutate}\NormalTok{(}\DataTypeTok{diff =}\NormalTok{ bmi_6yrs }\OperatorTok{-}\StringTok{ }\NormalTok{bmi_base)}
\end{Highlighting}
\end{Shaded}

\begin{verbatim}
## Parsed with column specification:
## cols(
##   ID = col_integer(),
##   BMI_base = col_double(),
##   BMI_6yrs = col_double()
## )
\end{verbatim}

\begin{Shaded}
\begin{Highlighting}[]
\NormalTok{n1=}\DecValTok{10}
\NormalTok{n2=}\DecValTok{10}
\KeywordTok{qf}\NormalTok{(}\FloatTok{0.975}\NormalTok{, n1}\OperatorTok{-}\DecValTok{1}\NormalTok{, n2}\OperatorTok{-}\DecValTok{1}\NormalTok{)}
\end{Highlighting}
\end{Shaded}

\begin{verbatim}
## [1] 4.025994
\end{verbatim}

\begin{Shaded}
\begin{Highlighting}[]
\CommentTok{#test equality for variances}
\KeywordTok{var.test}\NormalTok{(smoke_data}\OperatorTok{$}\NormalTok{diff, nonsmoke_data}\OperatorTok{$}\NormalTok{diff, }\DataTypeTok{alternative =} \StringTok{"two.sided"}\NormalTok{)}
\end{Highlighting}
\end{Shaded}

\begin{verbatim}
## 
##  F test to compare two variances
## 
## data:  smoke_data$diff and nonsmoke_data$diff
## F = 1.1627, num df = 9, denom df = 9, p-value = 0.826
## alternative hypothesis: true ratio of variances is not equal to 1
## 95 percent confidence interval:
##  0.2888038 4.6811133
## sample estimates:
## ratio of variances 
##           1.162722
\end{verbatim}

\[
H_{0}: \sigma_{1} ^{2} = \sigma_{2} ^{2}
\]

\[
H_{0}: \sigma_{1} ^{2} \neq  \sigma_{2} ^{2}
\]

\[
F= \frac{s_{1}^{2}}{s_{2}^{2}}\sim F_{n_{1}-1,n_{2}-1}=1.162722 
\]

\[
F_{n_{1}-1,n_{2}-1} = 4.025994
\]

\[
For\ F< F_{n_{1}-1,n_{2}-1},\ fail\ to\ reject\ H_{0},\ \sigma_{1} ^{2} = \sigma_{2} ^{2}
\]

\begin{Shaded}
\begin{Highlighting}[]
\KeywordTok{qt}\NormalTok{(}\FloatTok{0.975}\NormalTok{, n1}\OperatorTok{+}\NormalTok{n2}\OperatorTok{-}\DecValTok{2}\NormalTok{)}
\end{Highlighting}
\end{Shaded}

\begin{verbatim}
## [1] 2.100922
\end{verbatim}

\begin{Shaded}
\begin{Highlighting}[]
\KeywordTok{t.test}\NormalTok{(nonsmoke_data}\OperatorTok{$}\NormalTok{diff, smoke_data}\OperatorTok{$}\NormalTok{diff, }\DataTypeTok{var.equal =} \OtherTok{TRUE}\NormalTok{, }\DataTypeTok{paired =} \OtherTok{FALSE}\NormalTok{)}
\end{Highlighting}
\end{Shaded}

\begin{verbatim}
## 
##  Two Sample t-test
## 
## data:  nonsmoke_data$diff and smoke_data$diff
## t = 4.6228, df = 18, p-value = 0.0002114
## alternative hypothesis: true difference in means is not equal to 0
## 95 percent confidence interval:
##  2.678568 7.141432
## sample estimates:
## mean of x mean of y 
##      1.55     -3.36
\end{verbatim}

\[
H_{0} : \mu_{1} = \mu_{2}
\]

\[
H_{1} : \mu_{1} \neq \mu_{2}
\]

\[
s^{2} = \frac{(n_{1}-1)s_{1}^{2}+(n_{2}-1)s_{2}^{2}}{n_{1}+n_{2}-2}= 5.6405
\]

\[
t = \frac{\bar{X_{1}}-\bar{X_{2}}}{s\sqrt{(\frac{1}{n_{1}}+\frac{1}{n_{2}})}}=4.6228\\
\]

\[
t_{n_{1}+n_{2}-2, 1-\alpha /2}=2.100922
\]

\[
For\ \left | t \right |>t_{n_{1}+n_{2}-2, 1-\alpha /2},\ reject\ H_{0},\ \mu_{1} \neq \mu_{2}
\]

Intepretation: First, we use F-test to test the equality of variances.
The result shows that the variances of two groups are equal. Then we use
t-test to test the equality of mean. The result shows that the means of
BMI changes of two groups are different. So there is significant BMI
changes between women who quit smoking and women who never smoked.

\subsection{3. Show the corresponding 95\% CI associated with part 2.
Interpret it in the context of the
problem.}\label{show-the-corresponding-95-ci-associated-with-part-2.-interpret-it-in-the-context-of-the-problem.}

\[
(\bar{X_{1}}-\bar{X_{2}}) - t_{n_{1}+n_{2}-2, 1-\alpha /2}s\sqrt{1/n_{1}+1/n_{2}}\leq \mu\leq (\bar{X_{1}}-\bar{X_{2}}) + t_{n_{1}+n_{2}-2, 1-\alpha /2}s\sqrt{1/n_{1}+1/n_{2}}
\]

\begin{Shaded}
\begin{Highlighting}[]
\NormalTok{t =}\StringTok{ }\KeywordTok{qt}\NormalTok{(}\FloatTok{0.975}\NormalTok{, }\DecValTok{18}\NormalTok{)}
\NormalTok{s =}\StringTok{ }\KeywordTok{sqrt}\NormalTok{(}\FloatTok{25.15739}\NormalTok{)}
\NormalTok{CIL =}\StringTok{ }\FloatTok{28.86} \OperatorTok{-}\StringTok{ }\FloatTok{30.41} \OperatorTok{-}\StringTok{ }\NormalTok{(t }\OperatorTok{*}\StringTok{ }\NormalTok{s }\OperatorTok{*}\StringTok{ }\KeywordTok{sqrt}\NormalTok{(}\DecValTok{2}\OperatorTok{/}\DecValTok{10}\NormalTok{))}
\NormalTok{CIR =}\StringTok{ }\FloatTok{28.86} \OperatorTok{-}\StringTok{ }\FloatTok{30.41} \OperatorTok{+}\StringTok{ }\NormalTok{(t }\OperatorTok{*}\StringTok{ }\NormalTok{s }\OperatorTok{*}\StringTok{ }\KeywordTok{sqrt}\NormalTok{(}\DecValTok{2}\OperatorTok{/}\DecValTok{10}\NormalTok{))}
\end{Highlighting}
\end{Shaded}

\[
(\bar{X_{1}}-\bar{X_{2}}) - t_{n_{1}+n_{2}-2, 1-\alpha /2}s\sqrt{1/n_{1}+1/n_{2}} = 
-6.262569
\] \[
(\bar{X_{1}}-\bar{X_{2}}) + t_{n_{1}+n_{2}-2, 1-\alpha /2}s\sqrt{1/n_{1}+1/n_{2}} = 3.162569
\] \[
-6.262569 \leq \mu \leq 3.162569
\] Intepretation: The 95\% CI for these two samples are (-6.262569,
3.162569). This CI means that we are 95\% confidence that the true
population mean difference between women that quit smoking and women who
never smoked lies between the lower and upper limits of the interval.

\subsection{4. Suppose the researchers want to launch into a larger
study to prove that a difference does exist between the two groups with
respect to BMI
changes.}\label{suppose-the-researchers-want-to-launch-into-a-larger-study-to-prove-that-a-difference-does-exist-between-the-two-groups-with-respect-to-bmi-changes.}

\subsubsection{a. How would you design the new study? Comment on
elements of study design such as randomization, possible causes of bias
that should be avoided, etc.
(5p)}\label{a.-how-would-you-design-the-new-study-comment-on-elements-of-study-design-such-as-randomization-possible-causes-of-bias-that-should-be-avoided-etc.-5p}

For this new study, I would choose 50 women who never smoked and 50
women who quit smoking. To build the counterfactual, we should make sure
that these two groups are comparable, which means except exposure, other
conditions of women in each group should be the same (e.g health
condition, age). Then, recording the BMI of each group. The possible
bias in this study should be avoided is that 1) we should have
sufficient sample size. Greater sample size can better represent the
population. If the sample size is too small, the result might be
inaccurate; 2) make sure there is no loss to follow up.

\subsubsection{b. Calculate the sample size for the new study. Assuming
a two-sided test, create a table showing sample size estimates for 80\%
vs 90\% power, 2.5\% vs 5\% significance level, using the following
information: the true mean increase for smokers is 3.0 kg/m2, with a
standard deviation of 2.0 kg/m2; for never-smokers the true mean
increase is 1.7 kg/m2, with a standard deviation of 1.5 kg/m2. (R only
is allowed for calculations).
(5p)}\label{b.-calculate-the-sample-size-for-the-new-study.-assuming-a-two-sided-test-create-a-table-showing-sample-size-estimates-for-80-vs-90-power-2.5-vs-5-significance-level-using-the-following-information-the-true-mean-increase-for-smokers-is-3.0-kgm2-with-a-standard-deviation-of-2.0-kgm2-for-never-smokers-the-true-mean-increase-is-1.7-kgm2-with-a-standard-deviation-of-1.5-kgm2.-r-only-is-allowed-for-calculations.-5p}

\[
\Delta = 3-1.7=1.3
\]

\[
n = \frac{(z_{1-\beta} + z_{1-\alpha /2})^{2}(\sigma_{1}^{2}+\sigma_{2}^{2})}{\Delta ^{2}}
\]

\begin{longtable}[]{@{}lll@{}}
\toprule
Power & 0.8 & 0.9\tabularnewline
\midrule
\endhead
\(\alpha\) & &\tabularnewline
0.025 & 46 & 60\tabularnewline
0.05 & 38 & 50\tabularnewline
\bottomrule
\end{longtable}

\section{Problem 3}\label{problem-3}

A rehabilitation center is interested in examining the relationship
between physical status before therapy and the time (days) required in
physical therapy until successful rehabilitation. Records from patients
18-30 years old were collected and provided to you for statistical
analysis (data ``Knee.csv'').

Assuming that data are normally distributed, answer the questions below:

\subsection{1. Generate descriptive statistics for each group and
comment on the differences observed (R only).
(4p)}\label{generate-descriptive-statistics-for-each-group-and-comment-on-the-differences-observed-r-only.-4p}

\begin{Shaded}
\begin{Highlighting}[]
\NormalTok{knee_data =}\StringTok{ }\KeywordTok{read_csv}\NormalTok{(}\StringTok{"./Knee.csv"}\NormalTok{) }\OperatorTok
\StringTok{  }\NormalTok{janitor}\OperatorTok{::}\KeywordTok{clean_names}\NormalTok{()}
\end{Highlighting}
\end{Shaded}

\begin{verbatim}
## Parsed with column specification:
## cols(
##   Below = col_integer(),
##   Average = col_integer(),
##   Above = col_integer()
## )
\end{verbatim}

\begin{Shaded}
\begin{Highlighting}[]
\KeywordTok{summary}\NormalTok{(knee_data}\OperatorTok{$}\NormalTok{below)}
\end{Highlighting}
\end{Shaded}

\begin{verbatim}
##    Min. 1st Qu.  Median    Mean 3rd Qu.    Max.    NA's 
##      29      36      40      38      42      43       2
\end{verbatim}

\begin{Shaded}
\begin{Highlighting}[]
\KeywordTok{sd}\NormalTok{(knee_data}\OperatorTok{$}\NormalTok{below, }\DataTypeTok{na.rm =}\NormalTok{ T)}
\end{Highlighting}
\end{Shaded}

\begin{verbatim}
## [1] 5.477226
\end{verbatim}

\begin{Shaded}
\begin{Highlighting}[]
\KeywordTok{summary}\NormalTok{(knee_data}\OperatorTok{$}\NormalTok{average)}
\end{Highlighting}
\end{Shaded}

\begin{verbatim}
##    Min. 1st Qu.  Median    Mean 3rd Qu.    Max. 
##   28.00   30.25   32.00   33.00   35.00   39.00
\end{verbatim}

\begin{Shaded}
\begin{Highlighting}[]
\KeywordTok{sd}\NormalTok{(knee_data}\OperatorTok{$}\NormalTok{average, }\DataTypeTok{na.rm =}\NormalTok{ T)}
\end{Highlighting}
\end{Shaded}

\begin{verbatim}
## [1] 3.91578
\end{verbatim}

\begin{Shaded}
\begin{Highlighting}[]
\KeywordTok{summary}\NormalTok{(knee_data}\OperatorTok{$}\NormalTok{above)}
\end{Highlighting}
\end{Shaded}

\begin{verbatim}
##    Min. 1st Qu.  Median    Mean 3rd Qu.    Max.    NA's 
##   20.00   21.00   22.00   23.57   24.50   32.00       3
\end{verbatim}

\begin{Shaded}
\begin{Highlighting}[]
\KeywordTok{sd}\NormalTok{(knee_data}\OperatorTok{$}\NormalTok{above, }\DataTypeTok{na.rm =}\NormalTok{ T)}
\end{Highlighting}
\end{Shaded}

\begin{verbatim}
## [1] 4.197505
\end{verbatim}

\begin{longtable}[]{@{}llll@{}}
\toprule
Group & mean & sd & 3rd Qu.\tabularnewline
\midrule
\endhead
Below & 38 & 5.477226 & 42\tabularnewline
Average & 33 & 3.91578 & 35\tabularnewline
Above & 23.57 & 4.197505 & 24.5\tabularnewline
\bottomrule
\end{longtable}

As we can see above, the average days required in physical therapy until
successful rehabilitation is largest in Below group and smallest in
Above group. The number of 3rd Qu. is largest in the Below group and
smallest in the Above group. These two findings are conform with our
common sense. If the physical status before therapy is relatively
better, the days required should be relatively shorter.

\subsection{2. Using a type I error of 0.01, obtain the ANOVA table.
State the hypotheses, decision rule and conclusion (R only).
(5p)}\label{using-a-type-i-error-of-0.01-obtain-the-anova-table.-state-the-hypotheses-decision-rule-and-conclusion-r-only.-5p}

\begin{Shaded}
\begin{Highlighting}[]
\NormalTok{below <-}\StringTok{ }\NormalTok{knee_data}\OperatorTok{$}\NormalTok{below}
\NormalTok{average <-}\StringTok{ }\NormalTok{knee_data}\OperatorTok{$}\NormalTok{average}
\NormalTok{above <-}\StringTok{ }\NormalTok{knee_data}\OperatorTok{$}\NormalTok{above}

\NormalTok{knee_reshape <-}\StringTok{ }\KeywordTok{c}\NormalTok{(below, average, above)}
\NormalTok{ind<-}\KeywordTok{c}\NormalTok{(}\KeywordTok{rep}\NormalTok{(}\DecValTok{3}\NormalTok{,}\KeywordTok{length}\NormalTok{(below)),}\KeywordTok{rep}\NormalTok{(}\DecValTok{2}\NormalTok{,}\KeywordTok{length}\NormalTok{(average)),}\KeywordTok{rep}\NormalTok{(}\DecValTok{1}\NormalTok{,}\KeywordTok{length}\NormalTok{(above)))}
\NormalTok{new_data_knee <-}\StringTok{ }\KeywordTok{as.data.frame}\NormalTok{(}\KeywordTok{cbind}\NormalTok{(knee_reshape,ind))}

\NormalTok{res<-}\KeywordTok{lm}\NormalTok{(knee_reshape}\OperatorTok{~}\KeywordTok{factor}\NormalTok{(ind), }\DataTypeTok{data=}\NormalTok{new_data_knee)}
\KeywordTok{anova}\NormalTok{(res)}
\end{Highlighting}
\end{Shaded}

\begin{verbatim}
## Analysis of Variance Table
## 
## Response: knee_reshape
##             Df Sum Sq Mean Sq F value    Pr(>F)    
## factor(ind)  2 795.25  397.62   19.28 1.454e-05 ***
## Residuals   22 453.71   20.62                      
## ---
## Signif. codes:  0 '***' 0.001 '**' 0.01 '*' 0.05 '.' 0.1 ' ' 1
\end{verbatim}

\[
H_{0}:\ \mu_{1} = \mu_{2} = \mu_{3} =\ ...\ =\mu_{k} \\
H_{1}:\ at\ least\ two\ means\ are\ not\ equal \\
\] \[
F = \frac{Between\ SS/(k-1)}{With\ SS/(n-k)}\sim F_{k-1,n-k}\ distribution\ under\ H_{0}\\
F = 19.28
\]

\begin{Shaded}
\begin{Highlighting}[]
\KeywordTok{qf}\NormalTok{(}\DecValTok{1}\OperatorTok{-}\FloatTok{0.01}\NormalTok{, }\DecValTok{2}\NormalTok{, }\DecValTok{22}\NormalTok{)}
\end{Highlighting}
\end{Shaded}

\begin{verbatim}
## [1] 5.719022
\end{verbatim}

\[
F_{k-1,n-k,1-\alpha}=5.719022\\
F > F_{k-1,n-k,1-\alpha},\ reject\ H_{0}
\] \#\# 3. Based on your response in part 3, perform pairwise
comparisons with the appropriate adjustments (Bonferroni, Tukey, and
Dunnett -- `below average' as reference). Report your findings and
comment on the differences/similarities between these three methods (R
only). (5p)

\subsubsection{Bonferroni}\label{bonferroni}

\begin{Shaded}
\begin{Highlighting}[]
\KeywordTok{pairwise.t.test}\NormalTok{(new_data_knee}\OperatorTok{$}\NormalTok{knee_reshape, new_data_knee}\OperatorTok{$}\NormalTok{ind, }\DataTypeTok{p.adj=}\StringTok{'bonferroni'}\NormalTok{)}
\end{Highlighting}
\end{Shaded}

\begin{verbatim}
## 
##  Pairwise comparisons using t tests with pooled SD 
## 
## data:  new_data_knee$knee_reshape and new_data_knee$ind 
## 
##   1       2     
## 2 0.0011  -     
## 3 1.1e-05 0.0898
## 
## P value adjustment method: bonferroni
\end{verbatim}

\begin{Shaded}
\begin{Highlighting}[]
\KeywordTok{qt}\NormalTok{( }\DecValTok{1}\OperatorTok{-}\NormalTok{((}\FloatTok{0.01}\OperatorTok{/}\DecValTok{3}\NormalTok{)}\OperatorTok{/}\DecValTok{2}\NormalTok{), }\DecValTok{22}\NormalTok{ )}
\end{Highlighting}
\end{Shaded}

\begin{verbatim}
## [1] 3.290888
\end{verbatim}

As we can see, according to the bonferroni adjustment, there are no mean
differences between each group, which means: \[
\mu_{1}=\mu_{2}=\mu_{3}
\]

\subsubsection{Tukey}\label{tukey}

\begin{Shaded}
\begin{Highlighting}[]
\NormalTok{res1<-}\KeywordTok{aov}\NormalTok{(knee_reshape}\OperatorTok{~}\KeywordTok{factor}\NormalTok{(ind), }\DataTypeTok{data=}\NormalTok{new_data_knee)}
\KeywordTok{summary}\NormalTok{(res1)}
\end{Highlighting}
\end{Shaded}

\begin{verbatim}
##             Df Sum Sq Mean Sq F value   Pr(>F)    
## factor(ind)  2  795.2   397.6   19.28 1.45e-05 ***
## Residuals   22  453.7    20.6                     
## ---
## Signif. codes:  0 '***' 0.001 '**' 0.01 '*' 0.05 '.' 0.1 ' ' 1
## 5 observations deleted due to missingness
\end{verbatim}

\begin{Shaded}
\begin{Highlighting}[]
\KeywordTok{TukeyHSD}\NormalTok{(res1, }\DataTypeTok{conf.level =} \FloatTok{0.99}\NormalTok{)}
\end{Highlighting}
\end{Shaded}

\begin{verbatim}
##   Tukey multiple comparisons of means
##     99% family-wise confidence level
## 
## Fit: aov(formula = knee_reshape ~ factor(ind), data = new_data_knee)
## 
## $`factor(ind)`
##          diff       lwr      upr     p adj
## 2-1  9.428571  2.168498 16.68864 0.0010053
## 3-1 14.428571  6.803969 22.05317 0.0000102
## 3-2  5.000000 -1.988063 11.98806 0.0736833
\end{verbatim}

According to the Tukey method, we can see that the mean between below
and above and the mean between average and above are different. Tukey
method is less conservative than Bonferroni.

\subsubsection{Dunnett}\label{dunnett}

\begin{Shaded}
\begin{Highlighting}[]
\KeywordTok{library}\NormalTok{(DescTools)}
\end{Highlighting}
\end{Shaded}

\begin{Shaded}
\begin{Highlighting}[]
\NormalTok{x <-}\StringTok{ }\KeywordTok{c}\NormalTok{(}\DecValTok{29}\NormalTok{,}\DecValTok{42}\NormalTok{,}\DecValTok{38}\NormalTok{,}\DecValTok{40}\NormalTok{,}\DecValTok{43}\NormalTok{,}\DecValTok{40}\NormalTok{,}\DecValTok{30}\NormalTok{,}\DecValTok{42}\NormalTok{)}
\NormalTok{y <-}\StringTok{ }\KeywordTok{c}\NormalTok{(}\DecValTok{30}\NormalTok{,}\DecValTok{35}\NormalTok{,}\DecValTok{39}\NormalTok{,}\DecValTok{28}\NormalTok{,}\DecValTok{31}\NormalTok{,}\DecValTok{31}\NormalTok{,}\DecValTok{29}\NormalTok{,}\DecValTok{35}\NormalTok{,}\DecValTok{39}\NormalTok{,}\DecValTok{33}\NormalTok{)}
\NormalTok{z <-}\StringTok{ }\KeywordTok{c}\NormalTok{(}\DecValTok{26}\NormalTok{,}\DecValTok{32}\NormalTok{,}\DecValTok{21}\NormalTok{,}\DecValTok{20}\NormalTok{,}\DecValTok{23}\NormalTok{,}\DecValTok{22}\NormalTok{,}\DecValTok{21}\NormalTok{)}
\NormalTok{dunn_knee <-}\StringTok{ }\KeywordTok{c}\NormalTok{(x,y,z)}
\NormalTok{g <-}\StringTok{ }\KeywordTok{factor}\NormalTok{(}\KeywordTok{rep}\NormalTok{(}\DecValTok{1}\OperatorTok{:}\DecValTok{3}\NormalTok{, }\KeywordTok{c}\NormalTok{(}\DecValTok{8}\NormalTok{, }\DecValTok{10}\NormalTok{, }\DecValTok{7}\NormalTok{)),}
            \DataTypeTok{labels =} \KeywordTok{c}\NormalTok{(}\StringTok{"below"}\NormalTok{,}
                       \StringTok{"average"}\NormalTok{,}
                       \StringTok{"above"}\NormalTok{))}
\KeywordTok{DunnettTest}\NormalTok{(dunn_knee, g, }\DataTypeTok{control =} \StringTok{"above"}\NormalTok{, }\DataTypeTok{conf.level =} \FloatTok{0.99}\NormalTok{)}
\end{Highlighting}
\end{Shaded}

\begin{verbatim}
## 
##   Dunnett's test for comparing several treatments with a control :  
##     99% family-wise confidence level
## 
## $above
##                    diff   lwr.ci   upr.ci    pval    
## below-above   14.428571 7.173453 21.68369 6.9e-06 ***
## average-above  9.428571 2.520317 16.33683 0.00069 ***
## 
## ---
## Signif. codes:  0 '***' 0.001 '**' 0.01 '*' 0.05 '.' 0.1 ' ' 1
\end{verbatim}

According to the Dunnett method, we can see that the mean between below
and above group, and the mean between average and above group are
different. This conclusion is consistent with the result using Tukey's
method.

\subsection{4. Write a short paragraph summarizing your results as if
you were presenting to the rehabilitation center
director.(1p)}\label{write-a-short-paragraph-summarizing-your-results-as-if-you-were-presenting-to-the-rehabilitation-center-director.1p}

According to the analysis, we can conclude that for person who is in
better physical status, the average days required in physical therapy is
relatively less. However, the average days differences between below
group and average group is not significant. We can conclude that the
average days taken by patients in above group until rehabilitation is
much less than patients in average group's and below group's. In a more
conservative way, we might also says there is actually no significant
differences between the average days taken by patients in each group.

\section{Problem 4}\label{problem-4}

For this problem you will use the built-in R data called
``UCBAdmissions'' (library `datasets'), an example of sex bias in
admission practices. You are interested in comparing the proportions of
women vs men admitted at Berkeley (over all departments).

\subsection{1. Provide point estimates and 95\% CIs for the overall
proportions of men and women admitted at Berkeley. Briefly comment on
the values.
(5p)}\label{provide-point-estimates-and-95-cis-for-the-overall-proportions-of-men-and-women-admitted-at-berkeley.-briefly-comment-on-the-values.-5p}

\begin{Shaded}
\begin{Highlighting}[]
\KeywordTok{library}\NormalTok{(datasets)}
\NormalTok{ucb_ad =}\StringTok{ }\KeywordTok{as.data.frame}\NormalTok{(UCBAdmissions) }\OperatorTok
\StringTok{  }\NormalTok{janitor}\OperatorTok{::}\KeywordTok{clean_names}\NormalTok{()}
\end{Highlighting}
\end{Shaded}

\begin{Shaded}
\begin{Highlighting}[]
\NormalTok{ucb_women =}\StringTok{ }\NormalTok{ucb_ad }\OperatorTok
\StringTok{  }\KeywordTok{filter}\NormalTok{(gender }\OperatorTok{==}\StringTok{ "Female"}\NormalTok{)}
\NormalTok{ucb_men =}\StringTok{ }\NormalTok{ucb_ad }\OperatorTok
\StringTok{  }\KeywordTok{filter}\NormalTok{(gender }\OperatorTok{==}\StringTok{ "Male"}\NormalTok{)}

\NormalTok{ucb_admitted_women =}\StringTok{ }\NormalTok{ucb_women }\OperatorTok
\StringTok{  }\KeywordTok{filter}\NormalTok{(admit }\OperatorTok{==}\StringTok{ "Admitted"}\NormalTok{)}
\NormalTok{ucb_admitted_men =}\StringTok{ }\NormalTok{ucb_men }\OperatorTok
\StringTok{  }\KeywordTok{filter}\NormalTok{(admit }\OperatorTok{==}\StringTok{ "Admitted"}\NormalTok{)}

\NormalTok{X_men =}\StringTok{ }\KeywordTok{sum}\NormalTok{(ucb_admitted_men}\OperatorTok{$}\NormalTok{freq)}
\NormalTok{X_women =}\StringTok{ }\KeywordTok{sum}\NormalTok{(ucb_admitted_women}\OperatorTok{$}\NormalTok{freq)}

\NormalTok{n_men =}\StringTok{ }\KeywordTok{sum}\NormalTok{(ucb_men}\OperatorTok{$}\NormalTok{freq)}
\NormalTok{n_women =}\StringTok{ }\KeywordTok{sum}\NormalTok{(ucb_women}\OperatorTok{$}\NormalTok{freq)}

\NormalTok{p_hat_men =}\StringTok{ }\NormalTok{X_men}\OperatorTok{/}\NormalTok{n_men}
\NormalTok{p_hat_women =}\StringTok{ }\NormalTok{X_women}\OperatorTok{/}\NormalTok{n_women}
\end{Highlighting}
\end{Shaded}

\begin{Shaded}
\begin{Highlighting}[]
\CommentTok{# CI for men}
\NormalTok{CIL_men =}\StringTok{ }\NormalTok{p_hat_men }\OperatorTok{-}\StringTok{ }\NormalTok{(}\KeywordTok{qnorm}\NormalTok{(}\FloatTok{0.975}\NormalTok{) }\OperatorTok{*}\StringTok{ }\KeywordTok{sqrt}\NormalTok{(p_hat_men }\OperatorTok{*}\StringTok{ }\NormalTok{(}\DecValTok{1}\OperatorTok{-}\NormalTok{p_hat_men)}\OperatorTok{/}\NormalTok{n_men))}
\NormalTok{CIR_men =}\StringTok{ }\NormalTok{p_hat_men }\OperatorTok{+}\StringTok{ }\NormalTok{(}\KeywordTok{qnorm}\NormalTok{(}\FloatTok{0.975}\NormalTok{) }\OperatorTok{*}\StringTok{ }\KeywordTok{sqrt}\NormalTok{(p_hat_men }\OperatorTok{*}\StringTok{ }\NormalTok{(}\DecValTok{1}\OperatorTok{-}\NormalTok{p_hat_men)}\OperatorTok{/}\NormalTok{n_men))}

\CommentTok{#CI for women}
\NormalTok{CIL_women =}\StringTok{ }\NormalTok{p_hat_women }\OperatorTok{-}\StringTok{ }\NormalTok{(}\KeywordTok{qnorm}\NormalTok{(}\FloatTok{0.975}\NormalTok{) }\OperatorTok{*}\StringTok{ }\KeywordTok{sqrt}\NormalTok{(p_hat_women }\OperatorTok{*}\StringTok{ }\NormalTok{(}\DecValTok{1}\OperatorTok{-}\NormalTok{p_hat_women)}\OperatorTok{/}\NormalTok{n_women))}
\NormalTok{CIR_women =}\StringTok{ }\NormalTok{p_hat_women }\OperatorTok{+}\StringTok{ }\NormalTok{(}\KeywordTok{qnorm}\NormalTok{(}\FloatTok{0.975}\NormalTok{) }\OperatorTok{*}\StringTok{ }\KeywordTok{sqrt}\NormalTok{(p_hat_women }\OperatorTok{*}\StringTok{ }\NormalTok{(}\DecValTok{1}\OperatorTok{-}\NormalTok{p_hat_women)}\OperatorTok{/}\NormalTok{n_women))}
\end{Highlighting}
\end{Shaded}

The calculation of CI for a population proportion: \[
\hat{p}=\frac{\sum_{i=1}^{n}X_{i}}{n}=\frac{X}{n}
\]

\[
E(\hat{p}) = p
\]

\[
Var(\hat{p}) = \frac{p(1-p)}{n}
\] \[
\hat{p} = \bar{X} \sim N(p,\frac{p(1-p)}{n})
\] \[
(\hat{p} - z_{1-\alpha/2}\sqrt{\frac{\hat{p}(1-\hat{p})}{n}},\hat{p} + z_{1-\alpha/2}\sqrt{\frac{\hat{p}(1-\hat{p})}{n}})
\]

For men: \[
\hat{p}=\frac{1198}{2691}=0.4451877
\] \[
\hat{p} - z_{1-\alpha/2}\sqrt{\frac{\hat{p}(1-\hat{p})}{n}}=0.4264102
\] \[
\hat{p} + z_{1-\alpha/2}\sqrt{\frac{\hat{p}(1-\hat{p})}{n}}=0.4639651
\] So for men the CI is: \[
(0.4264102,0.4639651)
\]

For women:

\[
\hat{p}=\frac{557}{1835}=0.3035422
\]

\[
\hat{p} - z_{1-\alpha/2}\sqrt{\frac{\hat{p}(1-\hat{p})}{n}}=0.2825051
\] \[
\hat{p} + z_{1-\alpha/2}\sqrt{\frac{\hat{p}(1-\hat{p})}{n}}=0.3245794
\] So for women the CI is: \[
(0.2825051,0.3245794)
\] Comment: We are 95\% confidence that the true population proportion
of men admitted at Berkeley lies between 0.4264102 and 0.4639651, and
the true population proportions of women admitted at Berkeley lies
between 0.2825051 and 0.3245794. Simply judging from number, it seems
that the CI of women is lower than CI of men. But we don't know for sure
until we do the hypothesis test.

\subsection{\texorpdfstring{2. Perform a hypothesis test to assess if
the two proportions in 1) are significantly different. Report the
results including the test statistic and p-value and an overall
conclusion of your findings. This part should contain both `hand' and R
calculations. For the latter, feel free to use built-in functions or to
create your own.
(5p)}{2. Perform a hypothesis test to assess if the two proportions in 1) are significantly different. Report the results including the test statistic and p-value and an overall conclusion of your findings. This part should contain both hand and R calculations. For the latter, feel free to use built-in functions or to create your own. (5p)}}\label{perform-a-hypothesis-test-to-assess-if-the-two-proportions-in-1-are-significantly-different.-report-the-results-including-the-test-statistic-and-p-value-and-an-overall-conclusion-of-your-findings.-this-part-should-contain-both-hand-and-r-calculations.-for-the-latter-feel-free-to-use-built-in-functions-or-to-create-your-own.-5p}

\begin{Shaded}
\begin{Highlighting}[]
\NormalTok{two.proptest_norm <-}\StringTok{ }\ControlFlowTok{function}\NormalTok{(x1, x2, n1, n2, }\DataTypeTok{p=}\OtherTok{NULL}\NormalTok{, }\DataTypeTok{conf.level=}\FloatTok{0.95}\NormalTok{, }\DataTypeTok{alternative=}\StringTok{"less"}\NormalTok{) \{}

\CommentTok{# phat1, phat2 are observed proportions of each group}
\CommentTok{# n1, n2 are sample sizes of each group}
\CommentTok{# x1, x2 are admitted number of each group}
\CommentTok{# phat is the weighted average of the two sample proportions}
\CommentTok{# p.value is the hypothesis value}
  
\NormalTok{  z.stat <-}\StringTok{ }\OtherTok{NULL}
\NormalTok{  cint <-}\StringTok{ }\OtherTok{NULL}
\NormalTok{  p.val <-}\StringTok{ }\OtherTok{NULL}
\NormalTok{  phat1 <-}\StringTok{ }\NormalTok{x1}\OperatorTok{/}\NormalTok{n1}
\NormalTok{  phat2 <-}\StringTok{ }\NormalTok{x2}\OperatorTok{/}\NormalTok{n2}
\NormalTok{  qhat1 <-}\StringTok{ }\DecValTok{1} \OperatorTok{-}\StringTok{ }\NormalTok{phat1}
\NormalTok{  qhat2 <-}\StringTok{ }\DecValTok{1} \OperatorTok{-}\StringTok{ }\NormalTok{phat2}
\NormalTok{  phat <-}\StringTok{ }\NormalTok{(n1}\OperatorTok{*}\NormalTok{phat1 }\OperatorTok{+}\StringTok{ }\NormalTok{n2}\OperatorTok{*}\NormalTok{phat2)}\OperatorTok{/}\NormalTok{(n1}\OperatorTok{+}\NormalTok{n2)}
\NormalTok{  qhat <-}\StringTok{ }\DecValTok{1} \OperatorTok{-}\StringTok{ }\NormalTok{phat}
  
 
  \ControlFlowTok{if}\NormalTok{(}\KeywordTok{length}\NormalTok{(p) }\OperatorTok{>}\StringTok{ }\DecValTok{0}\NormalTok{) \{ }
\NormalTok{    SE.phat <-}\StringTok{ }\KeywordTok{sqrt}\NormalTok{(phat}\OperatorTok{*}\NormalTok{qhat}\OperatorTok{*}\NormalTok{((}\DecValTok{1}\OperatorTok{/}\NormalTok{n1)}\OperatorTok{+}\NormalTok{(}\DecValTok{1}\OperatorTok{/}\NormalTok{n2))) }
\NormalTok{    z.stat <-}\StringTok{ }\NormalTok{(phat1 }\OperatorTok{-}\StringTok{ }\NormalTok{phat2)}\OperatorTok{/}\NormalTok{SE.phat}
    \ControlFlowTok{if}\NormalTok{(z.stat}\OperatorTok{>}\DecValTok{0}\NormalTok{) \{}
\NormalTok{      p.val <-}\StringTok{ }\KeywordTok{pnorm}\NormalTok{(z.stat, }\DataTypeTok{lower.tail =} \OtherTok{FALSE}\NormalTok{)\}}
    \ControlFlowTok{if}\NormalTok{(z.stat}\OperatorTok{<}\DecValTok{0}\NormalTok{) \{}
\NormalTok{      p.val <-}\StringTok{ }\KeywordTok{pnorm}\NormalTok{(z.stat, }\DataTypeTok{lower.tail =} \OtherTok{TRUE}\NormalTok{)}
\NormalTok{    \}}
      \ControlFlowTok{if}\NormalTok{(alternative}\OperatorTok{==}\StringTok{"two.sided"}\NormalTok{) \{}
\NormalTok{        p.val <-}\StringTok{ }\NormalTok{p.val }\OperatorTok{*}\StringTok{ }\DecValTok{2}\NormalTok{\}}
    
      \ControlFlowTok{if}\NormalTok{(alternative}\OperatorTok{==}\StringTok{"greater"}\NormalTok{) \{}
\NormalTok{        p.val <-}\StringTok{ }\DecValTok{1} \OperatorTok{-}\StringTok{ }\NormalTok{p.val}
\NormalTok{      \}}
\NormalTok{    \} }\ControlFlowTok{else}\NormalTok{ \{}
    \CommentTok{# Construct a confidence interval   }
\NormalTok{      SE.phat <-}\StringTok{ }\KeywordTok{sqrt}\NormalTok{((phat1}\OperatorTok{*}\NormalTok{qhat1}\OperatorTok{/}\NormalTok{n1) }\OperatorTok{+}\StringTok{ }\NormalTok{(phat2}\OperatorTok{*}\NormalTok{qhat2}\OperatorTok{/}\NormalTok{n2))}
\NormalTok{    \}}
\NormalTok{    cint <-}\StringTok{ }\NormalTok{phat1}\OperatorTok{-}\NormalTok{phat2 }\OperatorTok{+}\StringTok{ }\KeywordTok{c}\NormalTok{(}\OperatorTok{-}\DecValTok{1}\OperatorTok{*}\NormalTok{((}\KeywordTok{qnorm}\NormalTok{(((}\DecValTok{1} \OperatorTok{-}\StringTok{ }\NormalTok{conf.level)}\OperatorTok{/}\DecValTok{2}\NormalTok{) }\OperatorTok{+}\StringTok{ }\NormalTok{conf.level))}\OperatorTok{*}\NormalTok{SE.phat),}
\NormalTok{                        ((}\KeywordTok{qnorm}\NormalTok{(((}\DecValTok{1} \OperatorTok{-}\StringTok{ }\NormalTok{conf.level)}\OperatorTok{/}\DecValTok{2}\NormalTok{) }\OperatorTok{+}\StringTok{ }\NormalTok{conf.level))}\OperatorTok{*}\NormalTok{SE.phat))}
  
  \KeywordTok{return}\NormalTok{(}\KeywordTok{list}\NormalTok{(}\DataTypeTok{estimate=}\NormalTok{phat1}\OperatorTok{-}\NormalTok{phat2, }\DataTypeTok{z.stat=}\NormalTok{z.stat, }\DataTypeTok{p.val=}\NormalTok{p.val, }\DataTypeTok{cint=}\NormalTok{cint))}
\NormalTok{\}}
\end{Highlighting}
\end{Shaded}

\begin{Shaded}
\begin{Highlighting}[]
\KeywordTok{two.proptest_norm}\NormalTok{(}\DataTypeTok{x1=}\DecValTok{1198}\NormalTok{, }\DataTypeTok{x2=}\DecValTok{557}\NormalTok{, }\DataTypeTok{n1=}\DecValTok{2691}\NormalTok{, }\DataTypeTok{n2=}\DecValTok{1835}\NormalTok{, }\DataTypeTok{p=}\DecValTok{0}\NormalTok{, }\DataTypeTok{conf.level=}\FloatTok{0.95}\NormalTok{, }\DataTypeTok{alternative =} \StringTok{"two.sided"}\NormalTok{)}
\end{Highlighting}
\end{Shaded}

\begin{verbatim}
## $estimate
## [1] 0.1416454
## 
## $z.stat
## [1] 9.602358
## 
## $p.val
## [1] 7.8136e-22
## 
## $cint
## [1] 0.1127338 0.1705571
\end{verbatim}

According to the result of two-sample binomial test for proportions, the
test statistic is 9.602358, p-value is 7.8136e-22, confidence interval
is (0.1127338, 0.1705571). Because the null hypothesis states that there
is no differences between the proportions of each group, so the
difference between two proportions should be 0. According to the
confidence interval, we can see that 0 is not included. In this case, we
should reject the null. We can double check this conclusion through
p-value. The p-value is 7.8136e-22 which is way less than 0.05. In
conclusion, the proportion of male admitted to UCB and the proportion of
female admitted to UCB are significantly different.


\end{document}
